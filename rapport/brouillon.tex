\documentclass[a4paper,10pt]{report}
\usepackage[utf8]{inputenc}
\usepackage[T1]{fontenc}
\usepackage[french]{babel}
\usepackage[top=1pt, bottom=1pt, left=50pt, right=50pt]{geometry}
\usepackage{amsmath}
\usepackage{color}
\usepackage{url}
\usepackage{listings}

\begin{document}

\title{Rapport codes correcteur}
\author{Cyril BIDAUD -- Corentin CHEPEAU -- Benjamin SEBILLE}
\date{\today} 

\maketitle


\tableofcontents

\chapter*{Exo}
G = $\begin{pmatrix}
   0 & 0 & 1\\
   0 & 1 & 0\\
   0 & 1 & 1\\
   1 & 0 & 0\\
   1 & 0 & 1\\
   1 & 1 & 0\\
   1 & 1 & 1 
\end{pmatrix}$

\chapter{Outils Mathématique utilisé}
	\section{Ensemble}
		\subsection{Corps}
			\subsubsection{Addition}

			\subsubsection{Multiplication}

		\subsection{Structure d'espace vectoriel}



	\section{Matrice}
\chapter{Algorithme détection d'erreur}
	\section{Code de Hamming}
		\subsection{$C(7,4)$}
		\lstset{language=C}
		\lstinputlisting{../hamming/myerror.c}

voir \url{http://agreg-maths.univ-rennes1.fr/documentation/docs/codes.pdf} \\


\section{programmer des procédures qui opère dans les ensembles quotients de polynômes (groupe de Galois), c'est à dire l'addition, la multiplication, algo d'Euclide, etc.}

Définition groupe de galois avec un polynome : \url{https://fr.wikipedia.org/wiki/Groupe_de_Galois#D.C3.A9finition} \\
Exemple groupe de galois sur des polynome : \url{https://fr.wikipedia.org/wiki/Groupe_de_Galois#Exemples} \\

Théorie de Galois et géométrie algébrique : \url{http://webusers.imj-prg.fr/~jan.nekovar/co/ln/gal/g.pdf} \\

Aide compréhension Groupe de Galois : \url{http://www.rennes.supelec.fr/ren/rd/scee/ftp/docs/corpsdegalois.pdf} \\
Aide compréhension Groupe de Galois 2 : \url{http://blogperso.univ-rennes1.fr/jeremy.le-borgne/public/introgalois.pdf} \\

Définitions et Exemples de Groupe en Algèbre Linéaire : \url{http://stephane.gonnord.org/PCSI/Algebre/STRUCTURES.PDF} \\

Algorithme d'Euclide (PGCD) : \url{https://fr.wikipedia.org/wiki/Algorithme_d%27Euclide#Fractions_continues} \\

Algèbres de Polynômes : \url{http://www.univ-orleans.fr/mapmo/membres/khaoula/enseignement/cours-algebre.pdf} \\

CDI : Mathématique pour informatique (chapitre code correcteur), Calcul avec des logiciels  \\
voir td licence : \url{http://download.tuxfamily.org/tehessinmath/les%20pdf/TDcodesLineaires.pdf} \\
+ calcul avec des logiciels libres \\

\section{construire un code linéaire de détection et correction d'erreurs (parité, Hamming et BCH)}

encode 7,4 hamming python \\
- 7 bits dont 4 bits pour le message et 3 bits de parité. \\
la somme de contrôle correspond aux 3 bits de parité, elle peut ensuite permettre la correction d'erreur {\centering{}\huge(\textbf{A TRAVAILLER})} \\

voir : \url{https://fr.wikipedia.org/wiki/Code_de_Hamming#Exemple_:_le_cas_binaire_de_longueur_quatre} \\


REGARDER \url{http://www.isima.fr/~vbarra/IMG/pdf/codes_correcteurs.pdf} \\


\section{Bonus: même  travail sur le code de Reed-Solomon}


chapitres intéressants : \\


STRUCTURE DU RAPPORT :
\begin{itemize} 
    \item les recherches documentaires      
    \item outils mathématiques
    \item polynômes
    \item hamming
    \item reed-solomon
\end{itemize}
 \url{https://www.overleaf.com/3867798jrgdqv}
 \end{document}